%  ___           _          _
% |_ _|_ __  ___| |_ ___ __(_)
%  | || '_ \/ _ \  _/ -_|_-< |
% |___| .__/\___/\__\___/__/_|
%     |_|


% Slide 3 % % % % % % % %
\begin{frame}[plain]{\alert{Ipotesi}}
\label{frm:ipotesi:1}

  \metroset{block=fill}
  \begin{block}{Ipotesi 1}
      Lo stile Landmark, come nell’ambiente reale, presenta una strategia per prove ed errori durante la
      ricerca Web, caratterizzata dal fatto di \alert{aprire più link}, \alert{rivisitare più pagine} e usare di più il \alert{bottone
      indietro} rispetto agli stili Route e Survey.
  \end{block}
  \vspace{0.5em}


  \metroset{block=fill}
  \begin{block}{Ipotesi 2}
      Lo stile Landmark impiega \alert{più tempo} per la navigazione e ottiene \alert{meno informazioni} utili rispetto agli
      stili Route e Survey.
  \end{block}
  \vspace{0.5em}


  \metroset{block=fill}
  \begin{block}{Ipotesi 3}
      Lo stile Landmark \alert{esplora di meno lo spazio Web} con il mouse e in maniera \alert{meno focalizzata e ottimale}
      rispetto agli stili Route e Survey, cosi come succede nell'ambiente reale.
  \end{block}

\end{frame}
